\documentclass{report}
\usepackage{polski}
\usepackage[utf8]{inputenc}
\usepackage[margin=1cm]{geometry}
\usepackage{siunitx}
\usepackage{amsmath}
\usepackage{textcomp}
\usepackage{multirow}

\begin{document}

\begin{center}
{\Large Projekt \textit{TFTP} na Sieci Komputerowe}

Artur Zubilewicz
\end{center}

\begin{enumerate}
\item \textbf{Opis i uruchomienie.}

Celem projektu było zaimplementowanie własnego klienta i serwera \textit{TFTP (Trivial File Transfer Protocol)} na podstawie dokumentów \textit{RFC 1350}, \textit{RFC 2347}, \textit{RFC 2348}, \textit{RFC 2349} i \textit{RFC 7440}.

Należało napisać serwer, który wysyłał pliki do klienta, a więc obsługiwane miało być polecenie \textit{RRQ (Read Request)}. Następnie klient obliczał hash \textit{md5} otrzymanego pliku i zwracał ją na output.

Aby uruchomić pobieranie należy najpierw uruchomić \textit{server.py} podając mu dwa argumenty pozycyjne \textit{directory} (folder, z którego mają być brane pliki do pobierania przez klienta) i \textit{port} (na którym ma nasłuchiwać przychodzących pakietów UDP). Można również włączyć symulację szumu poprzez dodanie flagi \textit{-{}-noise}, wówczas wysłanie pakietu powiedzie się z prawdopodobieństwem 90\%.

Następnie należy uruchomić \textit{client.py} podając mu dwa argumenty pozycyjne \textit{server} (nazwa serwera, z którego ma być pobrany plik, np. jego adres publiczny albo localhost dla obecnej maszyny) i \textit{filename} (nazwa pliku do pobrania). Można również dostarczyć argumenty modyfikujące jego działanie, jak np.:
\begin{itemize}
\item \textit{-{}-blocksize (RFC 2348)} - rozmiar pojedynczego bloku danych pliku przesyłanego w pakiecie TFTP; liczba całkowita dodatnia,
\item \textit{-{}-timeout (RFC 2349)} - czas, który musi minąć zanim klient i serwer przestaną czekać na pakiet TFTP i wyślą ponownie swój ostatni pakiet; liczba rzeczywista dodatnia,
\item \textit{-{}-windowsize (RFC 7440)} - liczba pakietów, które mogą być wysłane zanim nastąpi czekanie na ACK (acknowledgement) od klienta z informacją, że otrzymał pakiety; wartość całkowita dodatnia,
\item \textit{-{}-retries (opcja własna)}, pozwala ustalić liczbę timeoutów zanim klient jak i serwer poddadzą się przy kolejnych nieudanych czekaniach i zakończą działanie; liczba całkowita dodatnia.
\end{itemize}

\item \textbf{Implementacja.}

Kod symulatora został napisany w języku \textit{Python 3.6}. Implementacja składa się z  dwóch plików:
\begin{itemize}
\item \textit{client.py}
\item \textit{server.py}
\end{itemize}

Kod klienta składa się z metod, które pozwalają wysyłać pakiety $ACK$, $ERROR$ oraz $RRQ$ w zgodzie z $RFC\ 2347$, to jest z możliwością dodawania opcji.

Kod serwera składa się z metod, które pozwalają wysyłać pakiety $DATA$, $ERROR$ i $OACK$, również zgodnie z $RFC\ 2347$.

\item \textbf{Testy.}

Kod przetestowałem na swojej lokalnej maszynie poprzez utworzenie kilku plików o różnych rozmiarach i przesłanie ich za pomocą serwera i porównanie wynikowego hasha $md5$ z oryginalnym.

Pliki starałem się utworzyć tak, aby jak najniekorzystniej wpłynęły na działanie programów, tj. pozwalały zapętlać się numerom bloków $DATA$, sama długość pliku była długości podzielnej przez długość bloku, itp.

Dla przykładu, utworzyłem trzy pliki, pierwszy posiadający domyślny test, tj. \textit{echo "Zażółć gęślą jaźń!" $>$ test.txt} oraz dwa 'duże' pliki złożone z kopii znaku 'a': \textit{printf 'a\%.0s' \{1..33554432\} $>$ BIG32MiB.txt} i \textit{printf 'a\%.0s' \{1..67108864\} $>$ BIG64MiB.txt} (gdzie $67108864 = 2\cdot 33554432 = 2 \cdot 512 \cdot 256 \cdot 256$).

Uruchomienie serwera i klienta z domyślnymi ustawieniami jak również z przykładowymi flagami \textit{-{}-blocksize 123 -{}-timeout 2.0 -{}-windowsize 100 -{}-retries 5} zwracało te same wyniki, które pokrywały się z obliczonymi wcześniej wartościami haszy $md5$ za pomocą programu $md5sum$.

Testowałem również kod w przypadku 'szumu', tj. w sytuacji, gdy niektóre pakiety znikają na drodze między klientem i serwerem. Symulację takiego zachowania przeprowadziłem poprzes losowanie liczby z przedziału $[0..1]$ i jeżeli wylosowana liczba była mniejsza niż $0.5$ to nie wysyłałem pakietu.
\end{enumerate}
\end{document}

